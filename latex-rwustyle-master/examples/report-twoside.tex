\documentclass[twoside]{scrreprt}
\usepackage[ngerman]{babel}
\usepackage[head=true,foot=true,colorhead=true]{rwukoma}
\usepackage[pdfusetitle]{hyperref}
\usepackage{lipsum}

\title{Beispiel für ein mittellanges, doppelseitiges \LaTeX-Dokument}
\author{Stefan Gast}
\date{\today}

\begin{document}
	\maketitle
	\tableofcontents

	\chapter{Erstes Kapitel}

		Mit der \LaTeX-Klasse \verb?scrreprt? lassen sich längere Texte
		verfassen.
		Dieser Text hier ist für den doppelseitigen (Duplex) Druck
		vorgesehen.

		In den Standardeinstellungen verändert der \verb?rwukoma?-Style
		die Kopf- und Fußzeilen für die Klasse \verb?scrreprt? nicht
		und es werden die Standardeinstellungen aus der Klasse verwendet.

		Soll, so wie hier, dennoch das Logo und die Gesamtseitenzahl
		auf jeder regulären Seite erscheinen, so kann das über die
		Optionen \verb?head=true? und \verb?foot=true? angefordert werden.

		\section{Erster Abschnitt}

			\lipsum[1]

			\subsection{Erster Unterabschnitt}

				\lipsum[2]

				\subsubsection{Erster Unterunterabschnitt}

					\lipsum[2-5]

				\subsubsection{Zweiter Unterabschnitt}

					\lipsum[6-12]

	\chapter{Zweites Kapitel}

		Hier gibt es nichts mehr zu sehen.

		\section{Noch ein Abschnitt}

			\lipsum
\end{document}
