\documentclass{scrbook}
\usepackage[ngerman]{babel}
\usepackage[colorhead=true]{rwukoma}
\usepackage[pdfusetitle]{hyperref}
\usepackage{lipsum,caption}

\KOMAoption{listof}{totoc}

\title{Beispiel für ein \LaTeX-Buch}
\subtitle{Kann man auch machen}
\author{Stefan Gast}
\date{\today}

\begin{document}
	\maketitle
	\tableofcontents

	\part{Grundlagen}

	\chapter{Erstes Kapitel}

		Mit der \LaTeX-Klasse \verb?scrbook? lassen sich ganze Bücher
		verfassen.

		\section{Erster Abschnitt}

			Testweise ist in Abbildung~\ref{fig:logo} das RWU-Logo
			zu sehen. Abbildung~\ref{fig:latex} zeigt das \LaTeX-Logo
			in groß.

			\begin{figure}
				\centering
				\rwulogo[width=0.5\columnwidth]
				\caption{Das RWU-Logo}
				\label{fig:logo}
			\end{figure}

			\begin{figure}
				\centering
				\Huge{\LaTeX}
				\caption{Das \LaTeX-Logo}
				\label{fig:latex}
			\end{figure}

			\lipsum[1]

			\subsection{Erster Unterabschnitt}

				\lipsum[2]

				\subsubsection{Erster Unterunterabschnitt}

					\lipsum[2-5]

				\subsubsection{Zweiter Unterabschnitt}

					\lipsum[6-12]

	\chapter{Zweites Kapitel}

		Hier gibt es nichts mehr zu sehen.

		\section{Noch ein Abschnitt}

			\lipsum

	\part{Eigentlicher Inhalt}

	\chapter{Nächstes Kapitel}

		Ja, so viel Inhalt haben wir gar nicht.

		\section{Weiterer Abschnitt}

			\lipsum

	\appendix
	\part{Abbildungs- und Tabellenverzeichnisse}
	\listoffigures
\end{document}
